\documentclass[]{report}
\usepackage[utf8]{inputenc}


% Title Page
\title{Gruppennummer 16}
\author{Lena Trautmann\\Hanna Huber\\Andreas Cremer}


\begin{document}
\maketitle

%\begin{abstract}
%\end{abstract}

\begin{enumerate}
	\item Kovarianzmatrix
	\begin{enumerate}
		\item
		ourCov.m erwartet eine $d \times n$ Matrix und gibt die dazugehörige Kovarianzmatrix zurück.
		\item
		Hier werden die Kovarianzmatrizen für daten.mat berechnet. In $C_{11}$ steht die Varianz in der ersten Dimension. In $C_{22}$ steht die Varianz in der zweiten Dimension. In $C_{12}$ und $C_{21}$ steht die Kovarianz.\\
		data1 hat eine hohe Varianz in der ersten und eine geringe Varianz in der zweiten Dimension. Die Kovarianz ist gering, die Datenpunkte bilden ein schmales Band parallel zur x-Achse.\\
		data2 hat eine geringe Varianz in der ersten und eine hohe Varianz in der zweiten Dimension und ebenfalls eine geringe Kovarianz. Die Datenpunkte bilden ein schmales Band parallel zur y-Achse.\\
		data3 hat eine sehr hohe und eine deutlich niedrigere Varianz sowie eine hohe Kovarianz. Dies führt zu leicht ansteigenden Band.
		data4 hat hohe nahe beieinander liegende Varianzen und eine Kovarianz nahe beim Nullpunkt. Dies führt zu einer Punktwolke ohne erkennbare Ordnung.
	\end{enumerate}
	\item PCA
	\begin{enumerate}
		
		\item
		\item
		Der erste Eigenvektor gibt die Richtung der höchsten Varianz an. Weitere Eigenvektoren stehen jeweils orthogonal auf alle schon vorhanden Eigenvektoren und geben die Richtung der höchsten verbleibenden Varianz an. Im Plot sind die Eigenvektoren durch blaue Striche durch den Mittelwert gekennzeichnet.
		\item
		Die Eigenwerte zu den Eigenvektoren geben die Varianz in Richtung des jeweiligen Eigenvektors an. Im Plot sind sie durch die Länge der Eigenvektormarkierungen dargestellt. Sie ergeben aufaddiert die Gesamtvarianz.
		\item
		In die Berechnung von Varianz und Kovarianz fließt hier der Abstand der Datenpunkte vom Nullpunkt des verwendeten Koordinatensystems mit ein. Somit kann man keine sinnvollen Schlussfolgerungen mehr ziehen.
		
		
	\end{enumerate}
\end{enumerate}

\end{document}          
